% Created 2019-09-09 Mon 08:53
% Intended LaTeX compiler: pdflatex
\documentclass[11pt]{article}
\usepackage[utf8]{inputenc}
\usepackage[T1]{fontenc}
\usepackage{graphicx}
\usepackage{grffile}
\usepackage{longtable}
\usepackage{wrapfig}
\usepackage{rotating}
\usepackage[normalem]{ulem}
\usepackage{amsmath}
\usepackage{textcomp}
\usepackage{amssymb}
\usepackage{capt-of}
\usepackage{hyperref}
\usepackage[left=1in,right=1in,top=1in,bottom=1in]{geometry}
\usepackage[backend=biber,style=authoryear,date=year]{biblatex}
\addbibresource{/Users/shik544/Dropbox/references/library.bib}
\usepackage[noblocks]{authblk}

\author[1,*]{Alexey N. Shiklomanov <alexey.shiklomanov@pnnl.gov>}
\affil[1]{Joint Global Change Research Institute, Pacific Northwest National Laboratory, College Park, MD}

\author[2]{Elizabeth M. Cowdery <ecowdery@bu.edu>}
\affil[2]{Department of Earth \& Environment, Boston University, Boston, MA}

\author[3]{Michael Bahn <Michael.Bahn@uibk.ac.at>}
\affil[3]{Institute of Ecology, University of Innsbruck, 6020 Innsbruck, Austria}

\author[4]{Chaeho Byun <chaeho.byun@mail.mcgill.ca>}
\affil[4]{School of Civil and Environmental Engineering, Yonsei University, Seoul 03722, Korea}

\author[5]{Steven Jansen <steven.jansen@uni-ulm.de>}
\affil[5]{Institute of Systematic Botany and Ecology, Ulm University, Albert-Einstein-Allee 11, 89081, Ulm, Germany}

\author[6]{Koen Kramer <koen.kramer@wur.nl>}
\affil[6]{Department of Vegetation, Forest, and Landscape Ecology, Wageningen Environmental Research and Wageningen University, Wageningen, Gelderland, The Netherlands}

\author[7,8]{Vanessa Minden <vanessa.minden@uni-oldenburg.de>}
\affil[7]{Institute for Biology and Environmental Sciences, Carl von Ossietzky-University of Oldenburg, Carl von Ossietzky Str. 9-11, 26129 Oldenburg, Germany}
\affil[8]{Department of Biology, Ecology and Evolution, Vrije Universiteit Brussel, Pleinlaan 2, 1050 Brussels}

\author[9]{\"Ulo Niinemets <ylo.niinemets@emu.ee>}
\affil[9]{Institute of Agricultural and Environmental Sciences, Estonian University of Life Sciences, Kreutzwaldi 1, 51014 Tartu, Estonia}

\author[10]{Yusuke Onoda <yusuke.onoda@gmail.com>}
\affil[10]{Graduate School of Agriculture, Kyoto University, Kyoto, 605-8503, Japan}

\author[11]{Nadejda A. Soudzilovskaia <n.a.soudzilovskaia@cml.leidenuniv.nl>}
\affil[11]{Conservation Biology Department, Institute of Environmental Sciences, Leiden University, Leiden, The Netherlands}

\author[2]{Michael C. Dietze <dietze@bu.edu>}

\affil[*]{Corresponding author; Phone: (301) 314-6713; Fax: (301) 314-6719; Email: alexey.shiklomanov@pnnl.gov; Mail: 5825 University Research Ct., Office 3533, College Park, MD 20740}

\renewcommand\Authfont{\footnotesize}
\renewcommand\Affilfont{\scriptsize}

\usepackage{hanging}
\date{}
\title{Does the leaf economic spectrum hold within plant functional types? A Bayesian multivariate trait meta-analysis}
\hypersetup{
 pdfauthor={},
 pdftitle={Does the leaf economic spectrum hold within plant functional types? A Bayesian multivariate trait meta-analysis},
 pdfkeywords={},
 pdfsubject={},
 pdfcreator={Emacs 26.3 (Org mode 9.3)}, 
 pdflang={English}}
\begin{document}

\maketitle
\noindent \textbf{Journal:} Ecological Applications

\clearpage

\section*{Appendix S1: References for the original sources of the TRY trait data used in our analysis.}

\begin{hangparas}{1.5em}{1}\hypertarget{citeproc_bib_item_1}{Adler, P. B., D. G. Milchunas, W. K. Lauenroth, O. E. Sala, and I. C. Burke. 2004. Functional Traits of Graminoids in Semi-Arid Steppes: a Test of Grazing Histories. Journal of Applied Ecology 41:653–663.}

\hypertarget{citeproc_bib_item_2}{Adriaenssens, S. 2012. Dry deposition and canopy exchange for temperate tree species under high nitrogen deposition.}

\hypertarget{citeproc_bib_item_3}{Akhmetzhanova, A. A., N. A. Soudzilovskaia, V. G. Onipchenko, W. K. Cornwell, V. A. Agafonov, I. A. Selivanov, and J. H. C. Cornelissen. 2012. A Rediscovered Treasure: Mycorrhizal Intensity Database for 3000 Vascular Plant Species Across the Former Soviet Union. Ecology 93:689–690.}

\hypertarget{citeproc_bib_item_4}{Araujo, A. de, J. Ometto, A. Dolman, B. Kruijt, M. Waterloo, and J. Ehleringer. 2012. LBA-ECO CD-02 C and N Isotopes in Leaves and Atmospheric CO2, Amazonas, Brazil. ORNL Distributed Active Archive Center.}

\hypertarget{citeproc_bib_item_5}{Atkin, O. K., M. Schortemeyer, N. McFarlane, and J. R. Evans. 1999. The Response of Fast- and Slow-Growing Acacia Species To Elevated Atmospheric CO2: An Analysis of the Underlying Components of Relative Growth Rate. Oecologia 120:544–554.}

\hypertarget{citeproc_bib_item_6}{Auger, S., and B. Shipley. 2012. Inter-Specific and Intra-Specific Trait Variation Along Short Environmental Gradients in an Old-Growth Temperate Forest. Journal of Vegetation Science 24:419–428.}

\hypertarget{citeproc_bib_item_7}{Bakker, C., J. Rodenburg, and P. M. van Bodegom. 2005. Effects of Ca- and Fe-Rich Seepage on P Availability and Plant Performance in Calcareous Dune Soils. Plant and Soil 275:111–122.}

\hypertarget{citeproc_bib_item_8}{Bakker, C., P. M. Van Bodegom, H. J. M. Nelissen, W. H. O. Ernst, and R. Aerts. 2006. Plant Responses To Rising Water Tables and Nutrient Management in Calcareous Dune Slacks. Plant Ecology 185:19–28.}

\hypertarget{citeproc_bib_item_9}{Baraloto, C., C. E. T. Paine, L. Poorter, J. Beauchene, D. Bonal, A.-M. Domenach, B. Hérault, S. Patiño, J.-C. Roggy, and J. Chave. 2010. Decoupled Leaf and Stem Economics in Rain Forest Trees. Ecology Letters 13:1338–1347.}

\hypertarget{citeproc_bib_item_10}{Beckmann, M., M. Hock, H. Bruelheide, and A. Erfmeier. 2012. The Role of UV-B Radiation in the Invasion of Hieracium pilosella-A Comparison of German and New Zealand Plants. Environmental and Experimental Botany 75:173–180.}

\hypertarget{citeproc_bib_item_11}{Blonder, B., V. Buzzard, I. Simova, L. Sloat, B. Boyle, R. Lipson, B. Aguilar-Beaucage, A. Andrade, B. Barber, C. Barnes, and et al. 2012. The Leaf-Area Shrinkage Effect Can Bias Paleoclimate and Ecology Research. American Journal of Botany 99:1756–1763.}

\hypertarget{citeproc_bib_item_12}{Blonder, B., C. Violle, L. P. Bentley, and B. J. Enquist. 2010. Venation Networks and the Origin of the Leaf Economics Spectrum. Ecology Letters 14:91–100.}

\hypertarget{citeproc_bib_item_13}{Blonder, B., C. Violle, and B. J. Enquist. 2013. Assessing the Causes and Scales of the Leaf Economics Spectrum Using Venation Networks Inpopulus Tremuloides. Journal of Ecology 101:981–989.}

\hypertarget{citeproc_bib_item_14}{Bocanegra, K., F. Fernández, and J. Galvis. 2015. Grupos Funcionales De Árboles En Bosques Secundarios De La Región Bajo Calima (Buenaventura, Colombia). Boletín Científico. Centro de Museos. Museo de Historia Natural 19:17–40.}

\hypertarget{citeproc_bib_item_15}{Bodegom, P. M. van, B. K. Sorrell, A. Oosthoek, C. Bakker, and R. Aerts. 2008. Separating the Effects of Partial Submergence and Soil Oxygen Demand on Plant Physiology. Ecology 89:193–204.}

\hypertarget{citeproc_bib_item_16}{Bond-Lamberty, B., S. T. Gower, C. Wang, P. Cyr, and H. Veldhuis. 2006. Nitrogen Dynamics of a Boreal Black Spruce Wildfire Chronosequence. Biogeochemistry 81:1–16.}

\hypertarget{citeproc_bib_item_17}{Bond-Lamberty, B., C. Wang, S. T. Gower, and J. Norman. 2002a. Leaf Area Dynamics of a Boreal Black Spruce Fire Chronosequence. Tree Physiology 22:993–1001.}

\hypertarget{citeproc_bib_item_18}{Bond-Lamberty, B., C. Wang, and S. T. Gower. 2002b. Aboveground and Belowground Biomass and Sapwood Area Allometric Equations for Six Boreal Tree Species of Northern Manitoba. Canadian Journal of Forest Research 32:1441–1450.}

\hypertarget{citeproc_bib_item_19}{Bond-Lamberty, B., C. Wang, and S. T. Gower. 2004. Net Primary Production and Net Ecosystem Production of a Boreal Black Spruce Wildfire Chronosequence. Global Change Biology 10:473–487.}

\hypertarget{citeproc_bib_item_20}{Brown, K. A., D. F. B. Flynn, N. K. Abram, J. C. Ingram, S. E. Johnson, and P. Wright. 2011. Assessing Natural Resource Use By Forest-Reliant Communities in Madagascar Using Functional Diversity and Functional Redundancy Metrics. PLoS ONE 6:e24107.}

\hypertarget{citeproc_bib_item_21}{Burrascano, S., R. Copiz, E. Del Vico, S. Fagiani, E. Giarrizzo, M. Mei, A. Mortelliti, F. M. Sabatini, and C. Blasi. 2015. Wild Boar Rooting Intensity Determines Shifts in Understorey Composition and Functional Traits. Community Ecology 16:244–253.}

\hypertarget{citeproc_bib_item_22}{Butterfield, B. J., and J. M. Briggs. 2010. Regeneration Niche Differentiates Functional Strategies of Desert Woody Plant Species. Oecologia 165:477–487.}

\hypertarget{citeproc_bib_item_23}{Byun, C., S. de Blois, and J. Brisson. 2012. Plant Functional Group Identity and Diversity Determine Biotic Resistance To Invasion By an Exotic Grass. Journal of Ecology 101:128–139.}

\hypertarget{citeproc_bib_item_24}{Campbell, C., L. Atkinson, J. Zaragoza-Castells, M. Lundmark, O. Atkin, and V. Hurry. 2007. Acclimation of Photosynthesis and Respiration Is Asynchronous in Response To Changes in Temperature Regardless of Plant Functional Group. New Phytologist 176:375–389.}

\hypertarget{citeproc_bib_item_25}{Campetella, G., Z. Botta-Dukát, C. Wellstein, R. Canullo, S. Gatto, S. Chelli, L. Mucina, and S. Bartha. 2011. Patterns of Plant Trait-Environment Relationships Along a Forest Succession Chronosequence. Agriculture, Ecosystems \& Environment 145:38–48.}

\hypertarget{citeproc_bib_item_26}{Carswell, F. E., P. Meir, E. V. Wandelli, L. C. M. Bonates, B. Kruijt, E. M. Barbosa, A. D. Nobre, J. Grace, and P. G. Jarvis. 2000. Photosynthetic Capacity in a Central Amazonian Rain Forest. Tree Physiology 20:179–186.}

\hypertarget{citeproc_bib_item_27}{Cavender-Bares, J., A. Keen, and B. Miles. 2006. Phylogenetic Structure of Floridian Plant Communities Depends on Taxonomic and Spatial Scale. Ecology 87:S109–S122.}

\hypertarget{citeproc_bib_item_28}{Cerabolini, B. E. L., G. Brusa, R. M. Ceriani, R. De Andreis, A. Luzzaro, and S. Pierce. 2010. Can CSR Classification Be Generally Applied Outside Britain? Plant Ecology 210:253–261.}

\hypertarget{citeproc_bib_item_29}{Chambers, J. Q., E. S. Tribuzy, L. C. Toledo, B. F. Crispim, N. Higuchi, J. d. Santos, A. C. Araújo, B. Kruijt, A. D. Nobre, and S. E. Trumbore. 2004. Respiration From a Tropical Forest Ecosystem: Partitioning of Sources and Low Carbon Use Efficiency. Ecological Applications 14:72–88.}

\hypertarget{citeproc_bib_item_30}{Chen, Y., W. Han, L. Tang, Z. Tang, and J. Fang. 2011. Leaf Nitrogen and Phosphorus Concentrations of Woody Plants Differ in Responses To Climate, Soil and Plant Growth Form. Ecography 36:178–184.}

\hypertarget{citeproc_bib_item_31}{Choat, B., S. Jansen, T. J. Brodribb, H. Cochard, S. Delzon, R. Bhaskar, S. J. Bucci, T. S. Feild, S. M. Gleason, U. G. Hacke, and et al. 2012. Global Convergence in the Vulnerability of Forests To Drought. Nature 491:752–755.}

\hypertarget{citeproc_bib_item_32}{Cornelissen, J. H. C. 1996. An Experimental Comparison of Leaf Decomposition Rates in a Wide Range of Temperate Plant Species and Types. The Journal of Ecology 84:573.}

\hypertarget{citeproc_bib_item_33}{Cornelissen, J. H. C., P. C. Diez, and R. Hunt. 1996. Seedling Growth, Allocation and Leaf Attributes in a Wide Range of Woody Plant Species and Types. The Journal of Ecology 84:755.}

\hypertarget{citeproc_bib_item_34}{Cornelissen, J. H. C., H. M. Quested, D. Gwynn-jones, R. S. P. van Logtestijn, M. A. H. De Beus, A. Kondratchuk, T. V. Callaghan, and R. Aerts. 2004. Leaf Digestibility and Litter Decomposability Are Related in a Wide Range of Subarctic Plant Species and Types. Functional Ecology 18:779–786.}

\hypertarget{citeproc_bib_item_35}{Cornelissen, J., B. Cerabolini, P. Castro-Díez, P. Villar-Salvador, G. Montserrat-Martí, J. Puyravaud, M. Maestro, M. Werger, and R. Aerts. 2003. Functional Traits of Woody Plants: Correspondence of Species Rankings Between Field Adults and Laboratory-Grown Seedlings? Journal of Vegetation Science 14:311–322.}

\hypertarget{citeproc_bib_item_36}{Cornwell, W. K., J. H. C. Cornelissen, K. Amatangelo, E. Dorrepaal, V. T. Eviner, O. Godoy, S. E. Hobbie, B. Hoorens, H. Kurokawa, N. Pérez-Harguindeguy, and et al. 2008. Plant Species Traits Are the Predominant Control on Litter Decomposition Rates Within Biomes Worldwide. Ecology Letters 11:1065–1071.}

\hypertarget{citeproc_bib_item_37}{Craine, J. M., A. J. Elmore, M. P. M. Aidar, M. Bustamante, T. E. Dawson, E. A. Hobbie, A. Kahmen, M. C. Mack, K. K. McLauchlan, A. Michelsen, and et al. 2009. Global Patterns of Foliar Nitrogen Isotopes and Their Relationships With Climate, Mycorrhizal Fungi, Foliar Nutrient Concentrations, and Nitrogen Availability. New Phytologist 183:980–992.}

\hypertarget{citeproc_bib_item_38}{Craine, J. M., W. G. Lee, W. J. Bond, R. J. Williams, and L. C. Johnson. 2005. Environmental Constraints on a Global Relationship Among Leaf and Root Traits of Grasses. Ecology 86:12–19.}

\hypertarget{citeproc_bib_item_39}{Craine, J. M., J. B. Nippert, E. G. Towne, S. Tucker, S. W. Kembel, A. Skibbe, and K. K. McLauchlan. 2011. Functional Consequences of Climate Change-Induced Plant Species Loss in a Tallgrass Prairie. Oecologia 165:1109–1117.}

\hypertarget{citeproc_bib_item_40}{Craine, J. M., E. G. Towne, T. W. Ocheltree, and J. B. Nippert. 2012. Community Traitscape of Foliar Nitrogen Isotopes Reveals N Availability Patterns in a Tallgrass Prairie. Plant and Soil 356:395–403.}

\hypertarget{citeproc_bib_item_41}{Craven, D., D. Braden, M. Ashton, G. Berlyn, M. Wishnie, and D. Dent. 2007. Between and Within-Site Comparisons of Structural and Physiological Characteristics and Foliar Nutrient Content of 14 Tree Species At a Wet, Fertile Site and a Dry, Infertile Site in Panama. Forest Ecology and Management 238:335–346.}

\hypertarget{citeproc_bib_item_42}{Demey, A., J. Staelens, L. Baeten, P. Boeckx, M. Hermy, J. Kattge, and K. Verheyen. 2013. Nutrient Input From Hemiparasitic Litter Favors Plant Species With a Fast-Growth Strategy. Plant and Soil 371:53–66.}

\hypertarget{citeproc_bib_item_43}{Diaz, S., J. Hodgson, K. Thompson, M. Cabido, J. Cornelissen, A. Jalili, G. Montserrat-Martí, J. Grime, F. Zarrinkamar, Y. Asri, and et al. 2004. The Plant Traits That Drive Ecosystems: Evidence From Three Continents. Journal of Vegetation Science 15:295–304.}

\hypertarget{citeproc_bib_item_44}{Domingues, T. F., L. A. Martinelli, and J. R. Ehleringer. 2013. Seasonal Patterns of Leaf-Level Photosynthetic Gas Exchange in an Eastern Amazonian Rain Forest. Plant Ecology \& Diversity 7:189–203.}

\hypertarget{citeproc_bib_item_45}{Domingues, T. F., P. Meir, T. R. Feldpausch, G. Saiz, E. M. Veenendaal, F. Schrodt, M. Bird, G. Djagbletey, F. Hien, H. Compaore, and et al. 2010. Co-Limitation of Photosynthetic Capacity By Nitrogen and Phosphorus in West Africa Woodlands. Plant, Cell \& Environment 33:959–980.}

\hypertarget{citeproc_bib_item_46}{Fitter, A. H., and H. J. Peat. 1994. The Ecological Flora Database. The Journal of Ecology 82:415.}

\hypertarget{citeproc_bib_item_47}{Fonseca, C. R., J. M. Overton, B. Collins, and M. Westoby. 2000. Shifts in Trait-Combinations Along Rainfall and Phosphorus Gradients. Journal of Ecology 88:964–977.}

\hypertarget{citeproc_bib_item_48}{Frenette-Dussault, C., B. Shipley, J.-F. Léger, D. Meziane, and Y. Hingrat. 2011. Functional Structure of an Arid Steppe Plant Community Reveals Similarities With Grime’s C-S-R Theory. Journal of Vegetation Science 23:208–222.}

\hypertarget{citeproc_bib_item_49}{Fyllas, N. M., S. Patiño, T. R. Baker, G. Bielefeld Nardoto, L. A. Martinelli, C. A. Quesada, R. Paiva, M. Schwarz, V. Horna, L. M. Mercado, and et al. 2009. Basin-Wide Variations in Foliar Properties of Amazonian Forest: Phylogeny, Soils and Climate. Biogeosciences 6:2677–2708.}

\hypertarget{citeproc_bib_item_50}{Gallagher, R. V., and M. R. Leishman. 2012. A Global Analysis of Trait Variation and Evolution in Climbing Plants. Journal of Biogeography 39:1757–1771.}

\hypertarget{citeproc_bib_item_51}{Garnier, E., S. Lavorel, P. Ansquer, H. Castro, P. Cruz, J. Dolezal, O. Eriksson, C. Fortunel, H. Freitas, C. Golodets, and et al. 2007. Assessing the Effects of Land-Use Change on Plant Traits, Communities and Ecosystem Functioning in Grasslands: a Standardized Methodology and Lessons From an Application To 11 European Sites. Annals of Botany 99:967–985.}

\hypertarget{citeproc_bib_item_52}{Givnish, T. J., R. A. Montgomery, and G. Goldstein. 2004. Adaptive Radiation of Photosynthetic Physiology in the Hawaiian Lobeliads: Light Regimes, Static Light Responses, and Whole-Plant Compensation Points. American Journal of Botany 91:228–246.}

\hypertarget{citeproc_bib_item_53}{Guerin, G. R., H. Wen, and A. J. Lowe. 2012. Leaf Morphology Shift Linked To Climate Change. Biology Letters 8:882–886.}

\hypertarget{citeproc_bib_item_54}{Gutiérrez, A. G., and A. Huth. 2012. Successional Stages of Primary Temperate Rainforests of Chiloé Island, Chile. Perspectives in Plant Ecology, Evolution and Systematics 14:243–256.}

\hypertarget{citeproc_bib_item_55}{Guy, A. L., J. M. Mischkolz, and E. G. Lamb. 2013. Limited Effects of Simulated Acidic Deposition on Seedling Survivorship and Root Morphology of Endemic Plant Taxa of the Athabasca Sand Dunes in Well-Watered Greenhouse Trials. Botany 91:176–181.}

\hypertarget{citeproc_bib_item_56}{Han, W., J. Fang, D. Guo, and Y. Zhang. 2005. Leaf Nitrogen and Phosphorus Stoichiometry Across 753 Terrestrial Plant Species in China. New Phytologist 168:377–385.}

\hypertarget{citeproc_bib_item_57}{Hickler, T. 1999. Plant Functional Types and community characteristics along environmental gradients on Öland’s Great Alvar. University of Lund, Lund, Sweden.}

\hypertarget{citeproc_bib_item_58}{Kattge, J., W. Knorr, T. Raddatz, and C. Wirth. 2009. Quantifying Photosynthetic Capacity and Its Relationship To Leaf Nitrogen Content for Global-Scale Terrestrial Biosphere Models. Global Change Biology 15:976–991.}

\hypertarget{citeproc_bib_item_59}{Kazakou, E., D. Vile, B. Shipley, C. Gallet, and E. Garnier. 2006. Co-Variations in Litter Decomposition, Leaf Traits and Plant Growth in Species From a Mediterranean Old-Field Succession. Functional Ecology 20:21–30.}

\hypertarget{citeproc_bib_item_60}{Kerkhoff, A. J., W. F. Fagan, J. J. Elser, and B. J. Enquist. 2006. Phylogenetic and Growth Form Variation in the Scaling of Nitrogen and Phosphorus in the Seed Plants. The American Naturalist 168:E103–E122.}

\hypertarget{citeproc_bib_item_61}{Kichenin, E., D. A. Wardle, D. A. Peltzer, C. W. Morse, and G. T. Freschet. 2013. Contrasting Effects of Plant Inter- and Intraspecific Variation on Community-Level Trait Measures Along an Environmental Gradient. Functional Ecology 27:1254–1261.}

\hypertarget{citeproc_bib_item_62}{Kisel, Y., A. C. Moreno-Letelier, D. Bogarín, M. P. Powell, M. W. Chase, and T. G. Barraclough. 2012. Testing the Link Between Population Genetic Differentiation and Clade Diversification in Costa Rican Orchids. Evolution 66:3035–3052.}

\hypertarget{citeproc_bib_item_63}{Kleyer, M., R. Bekker, I. Knevel, J. Bakker, K. Thompson, M. Sonnenschein, P. Poschlod, J. van Groenendael, L. Klimevs, J. Klimevsová, and et al. 2008. The LEDA Traitbase: a Database of Life-History Traits of the Northwest European Flora. Journal of Ecology 96:1266–1274.}

\hypertarget{citeproc_bib_item_64}{Kraft, N. J. B., R. Valencia, and D. D. Ackerly. 2008. Functional Traits and Niche-Based Tree Community Assembly in an Amazonian Forest. Science 322:580–582.}

\hypertarget{citeproc_bib_item_65}{Laughlin, D. C., P. Z. Fulé, D. W. Huffman, J. Crouse, and E. Laliberté. 2011. Climatic Constraints on Trait-Based Forest Assembly. Journal of Ecology 99:1489–1499.}

\hypertarget{citeproc_bib_item_66}{Loranger, J., S. T. Meyer, B. Shipley, J. Kattge, H. Loranger, C. Roscher, and W. W. Weisser. 2012. Predicting Invertebrate Herbivory From Plant Traits: Evidence From 51 Grassland Species in Experimental Monocultures. Ecology 93:2674–2682.}

\hypertarget{citeproc_bib_item_67}{Louault, F., V. Pillar, J. Aufrère, E. Garnier, and J.-F. Soussana. 2005. Plant Traits and Functional Types in Response To Reduced Disturbance in a Semi-Natural Grassland. Journal of Vegetation Science 16:151–160.}

\hypertarget{citeproc_bib_item_68}{Loveys, B. R., L. J. Atkinson, D. J. Sherlock, R. L. Roberts, A. H. Fitter, and O. K. Atkin. 2003. Thermal Acclimation of Leaf and Root Respiration: an Investigation Comparing Inherently Fast- and Slow-Growing Plant Species. Global Change Biology 9:895–910.}

\hypertarget{citeproc_bib_item_69}{Manzoni, S., G. Vico, S. Palmroth, A. Porporato, and G. Katul. 2013. Optimization of Stomatal Conductance for Maximum Carbon Gain Under Dynamic Soil Moisture. Advances in Water Resources 62:90–105.}

\hypertarget{citeproc_bib_item_70}{Medlyn, B. E., F. .-W. Badeck, D. G. G. De Pury, C. V. M. Barton, M. Broadmeadow, R. Ceulemans, P. De Angelis, M. Forstreuter, M. E. Jach, S. Kellomäki, and et al. 1999. Effects of Elevated [CO2] on Photosynthesis in European Forest Species: a Meta-Analysis of Model Parameters. Plant, Cell \& Environment 22:1475–1495.}

\hypertarget{citeproc_bib_item_71}{Meir, P., B. Kruijt, M. Broadmeadow, E. Barbosa, O. Kull, F. Carswell, A. Nobre, and P. G. Jarvis. 2002. Acclimation of Photosynthetic Capacity To Irradiance in Tree Canopies in Relation To Leaf Nitrogen Concentration and Leaf Mass Per Unit Area. Plant, Cell and Environment 25:343–357.}

\hypertarget{citeproc_bib_item_72}{Meir, P., P. E. Levy, J. Grace, and P. G. Jarvis. 2007. Photosynthetic Parameters From Two Contrasting Woody Vegetation Types in West Africa. Plant Ecology 192:277–287.}

\hypertarget{citeproc_bib_item_73}{Messier, J., B. J. McGill, and M. J. Lechowicz. 2010. How Do Traits Vary Across Ecological Scales? a Case for Trait-Based Ecology. Ecology Letters 13:838–848.}

\hypertarget{citeproc_bib_item_74}{Meziane, D., and B. Shipley. 1999. Interacting Determinants of Specific Leaf Area in 22 Herbaceous Species: Effects of Irradiance and Nutrient Availability. Plant, Cell \& Environment 22:447–459.}

\hypertarget{citeproc_bib_item_75}{Milla, R., and P. B. Reich. 2011. Multi-Trait Interactions, Not Phylogeny, Fine-Tune Leaf Size Reduction With Increasing Altitude. Annals of Botany 107:455–465.}

\hypertarget{citeproc_bib_item_76}{Minden, V., S. Andratschke, J. Spalke, H. Timmermann, and M. Kleyer. 2012. Plant Trait-Environment Relationships in Salt Marshes: Deviations From Predictions By Ecological Concepts. Perspectives in Plant Ecology, Evolution and Systematics 14:183–192.}

\hypertarget{citeproc_bib_item_77}{Minden, V., and M. Kleyer. 2011. Testing the Effect-Response Framework: Key Response and Effect Traits Determining Above-Ground Biomass of Salt Marshes. Journal of Vegetation Science 22:387–401.}

\hypertarget{citeproc_bib_item_78}{Müller, S. C., G. E. Overbeck, J. Pfadenhauer, and V. D. Pillar. 2006. Plant Functional Types of Woody Species Related To Fire Disturbance in Forest-Grassland Ecotones. Plant Ecology 189:1–14.}

\hypertarget{citeproc_bib_item_79}{Niinemets, Ü. 2001. Global-Scale Climatic Controls of Leaf Dry Mass Per Area, Density, and Thickness in Trees and Shrubs. Ecology 82:453–469.}

\hypertarget{citeproc_bib_item_80}{Ogaya, R., and J. Peñuelas. 2003. Comparative Field Study of Quercus Ilex and Phillyrea Latifolia: Photosynthetic Response To Experimental Drought Conditions. Environmental and Experimental Botany 50:137–148.}

\hypertarget{citeproc_bib_item_81}{Onoda, Y., M. Westoby, P. B. Adler, A. M. F. Choong, F. J. Clissold, J. H. C. Cornelissen, S. Díaz, N. J. Dominy, A. Elgart, L. Enrico, and et al. 2011. Global Patterns of Leaf Mechanical Properties. Ecology Letters 14:301–312.}

\hypertarget{citeproc_bib_item_82}{Ordoñez, J. C., P. M. van Bodegom, Witte Jan‐Philip M., R. P. Bartholomeus, J. R. van Hal, and R. Aerts. 2010. Plant Strategies in Relation To Resource Supply in Mesic To Wet Environments: Does Theory Mirror Nature? The American Naturalist 175:225–239.}

\hypertarget{citeproc_bib_item_83}{Pahl, A. T., J. Kollmann, A. Mayer, and S. Haider. 2013. No Evidence for Local Adaptation in an Invasive Alien Plant: Field and Greenhouse Experiments Tracing a Colonization Sequence. Annals of Botany 112:1921–1930.}

\hypertarget{citeproc_bib_item_84}{Peco, B., I. de Pablos, J. Traba, and C. Levassor. 2005. The Effect of Grazing Abandonment on Species Composition and Functional Traits: the Case of Dehesa Grasslands. Basic and Applied Ecology 6:175–183.}

\hypertarget{citeproc_bib_item_85}{Penuelas, J., J. Sardans, J. Llusià, S. M. Owen, J. Carnicer, T. W. Giambelluca, E. L. Rezende, M. Waite, and Ü. Niinemets. 2009. Faster Returns on “Leaf Economics” and Different Biogeochemical Niche in Invasive Compared With Native Plant Species. Global Change Biology 16:2171–2185.}

\hypertarget{citeproc_bib_item_86}{Pierce, S., G. Brusa, M. Sartori, and B. E. L. Cerabolini. 2012. Combined Use of Leaf Size and Economics Traits Allows Direct Comparison of Hydrophyte and Terrestrial Herbaceous Adaptive Strategies. Annals of Botany 109:1047–1053.}

\hypertarget{citeproc_bib_item_87}{Pierce, S., G. Brusa, I. Vagge, and B. E. L. Cerabolini. 2013. Allocating Csr Plant Functional Types: the Use of Leaf Economics and Size Traits To Classify Woody and Herbaceous Vascular Plants. Functional Ecology 27:1002–1010.}

\hypertarget{citeproc_bib_item_88}{Pierce, S., A. Luzzaro, M. Caccianiga, R. M. Ceriani, and B. Cerabolini. 2007. Disturbance Is the Principal $alpha$-scale Filter Determining Niche Differentiation, Coexistence and Biodiversity in an Alpine Community. Journal of Ecology 95:698–706.}

\hypertarget{citeproc_bib_item_89}{Pillar, V. D., and E. E. Sosinski. 2003. An Improved Method for Searching Plant Functional Types By Numerical Analysis. Journal of Vegetation Science 14:323–332.}

\hypertarget{citeproc_bib_item_90}{Poorter, H., Ü. Niinemets, L. Poorter, I. J. Wright, and R. Villar. 2009. Causes and Consequences of Variation in Leaf Mass Per Area (LMA): a Meta-Analysis. New Phytologist 182:565–588.}

\hypertarget{citeproc_bib_item_91}{Powers, J. S., and P. Tiffin. 2010. Plant Functional Type Classifications in Tropical Dry Forests in Costa Rica: Leaf Habit Versus Taxonomic Approaches. Functional Ecology 24:927–936.}

\hypertarget{citeproc_bib_item_92}{Prentice, I. C., T. Meng, H. Wang, S. P. Harrison, J. Ni, and G. Wang. 2010. Evidence of a Universal Scaling Relationship for Leaf CO2 Drawdown Along an Aridity Gradient. New Phytologist 190:169–180.}

\hypertarget{citeproc_bib_item_93}{Preston, K. A., W. K. Cornwell, and J. L. DeNoyer. 2006. Wood Density and Vessel Traits As Distinct Correlates of Ecological Strategy in 51 California Coast Range Angiosperms. New Phytologist 170:807–818.}

\hypertarget{citeproc_bib_item_94}{Price, C. A., and B. J. Enquist. 2007. Scaling Mass and Morphology in Leaves: an Extension of the Wbe Model. Ecology 88:1132–1141.}

\hypertarget{citeproc_bib_item_95}{Pyankov, V. I., A. V. Kondratchuk, and B. Shipley. 1999. Leaf Structure and Specific Leaf Mass: the Alpine Desert Plants of the Eastern Pamirs, Tadjikistan. New Phytologist 143:131–142.}

\hypertarget{citeproc_bib_item_96}{Quested, H. M., J. H. C. Cornelissen, M. C. Press, T. V. Callaghan, R. Aerts, F. Trosien, P. Riemann, D. Gwynn-Jones, A. Kondratchuk, and S. E. Jonasson. 2003. Decomposition of Sub-Arctic Plants With Differing Nitrogen Economies: a Functional Role for Hemiparasites. Ecology 84:3209–3221.}

\hypertarget{citeproc_bib_item_97}{Reich, P. B., M. G. Tjoelker, K. S. Pregitzer, I. J. Wright, J. Oleksyn, and J.-L. Machado. 2008. Scaling of Respiration To Nitrogen in Leaves, Stems and Roots of Higher Land Plants. Ecology Letters 11:793–801.}

\hypertarget{citeproc_bib_item_98}{Rüger, N., A. Huth, S. P. Hubbell, and R. Condit. 2009. Response of Recruitment To Light Availability Across a Tropical Lowland Rain Forest Community. Journal of Ecology 97:1360–1368.}

\hypertarget{citeproc_bib_item_99}{Rüger, N., A. Huth, S. P. Hubbell, and R. Condit. 2011. Determinants of Mortality Across a Tropical Lowland Rainforest Community. Oikos 120:1047–1056.}

\hypertarget{citeproc_bib_item_100}{Sandel, B., J. D. Corbin, and M. Krupa. 2011. Using Plant Functional Traits To Guide Restoration: a Case Study in California Coastal Grassland. Ecosphere 2:art23.}

\hypertarget{citeproc_bib_item_101}{Schererlorenzen, M., E. Schulze, A. Don, J. Schumacher, and E. Weller. 2007. Exploring the Functional Significance of Forest Diversity: a New Long-Term Experiment With Temperate Tree Species (BIOTREE). Perspectives in Plant Ecology, Evolution and Systematics 9:53–70.}

\hypertarget{citeproc_bib_item_102}{Schweingruber, F., and W. Landolt. 2005. The xylem database.}

\hypertarget{citeproc_bib_item_103}{Shiodera, S., J. S. Rahajoe, and T. Kohyama. 2008. Variation in Longevity and Traits of Leaves Among Co-Occurring Understorey Plants in a Tropical Montane Forest. Journal of Tropical Ecology 24:121–133.}

\hypertarget{citeproc_bib_item_104}{Shipley, B. 1995. Structured Interspecific Determinants of Specific Leaf Area in 34 Species of Herbaceous Angiosperms. Functional Ecology 9:312.}

\hypertarget{citeproc_bib_item_105}{Shipley, B. 2002. Trade-Offs Between Net Assimilation Rate and Specific Leaf Area in Determining Relative Growth Rate: Relationship With Daily Irradiance. Functional Ecology 16:682–689.}

\hypertarget{citeproc_bib_item_106}{Shipley, B., and M. J. Lechowicz. 2000. The Functional Co-Ordination of Leaf Morphology, Nitrogen Concentration, and Gas Exchange In40 Wetland Species. Écoscience 7:183–194.}

\hypertarget{citeproc_bib_item_107}{Shipley, B., and T.-T. Vu. 2002. Dry Matter Content As a Measure of Dry Matter Concentration in Plants and Their Parts. New Phytologist 153:359–364.}

\hypertarget{citeproc_bib_item_108}{Spasojevic, M. J., and K. N. Suding. 2012. Inferring Community Assembly Mechanisms From Functional Diversity Patterns: the Importance of Multiple Assembly Processes. Journal of Ecology 100:652–661.}

\hypertarget{citeproc_bib_item_109}{Swaine, E. 2007. Ecological and evolutionary drivers of plant community assembly in a Bornean rain forest. University of Aberdeen, Aberdeen, Scotland, UK.}

\hypertarget{citeproc_bib_item_110}{Tucker, S. S., J. M. Craine, and J. B. Nippert. 2011. Physiological Drought Tolerance and the Structuring of Tallgrass Prairie Assemblages. Ecosphere 2:art48.}

\hypertarget{citeproc_bib_item_111}{Veihmeyer, F. J. 1956. Soil Moisture. Pflanze und Wasser / Water Relations of Plants:64–123.}

\hypertarget{citeproc_bib_item_112}{Vergutz, L., S. Manzoni, A. Porporato, R. Novais, and R. Jackson. 2012. A Global Database of Carbon and Nutrient Concentrations of Green and Senesced Leaves.}

\hypertarget{citeproc_bib_item_113}{Vile, D. 2005. Significations fonctionnelle et ecologique des traits des especes vegetales: Exemple dans une succession post-cultural Mediterraneenne et generalisations.}

\hypertarget{citeproc_bib_item_114}{Von Holle, B., and D. Simberloff. 2004. Testing Fox’s Assembly Rule: Does Plant Invasion Depend on Recipient Community Structure? Oikos 105:551–563.}

\hypertarget{citeproc_bib_item_115}{Weedon, J. T., W. K. Cornwell, J. H. Cornelissen, A. E. Zanne, C. Wirth, and D. A. Coomes. 2009. Global Meta-Analysis of Wood Decomposition Rates: a Role for Trait Variation Among Tree Species? Ecology Letters 12:45–56.}

\hypertarget{citeproc_bib_item_116}{Williams, M., Y. Shimabokuro, and E. Rastetter. 2012. LBA-ECO CD-09 Soil and Vegetation Characteristics, Tapajos National Forest, Brazil.}

\hypertarget{citeproc_bib_item_117}{Willis, C. G., M. Halina, C. Lehman, P. B. Reich, A. Keen, S. McCarthy, and J. Cavender-Bares. 2009. Phylogenetic Community Structure in Minnesota Oak Savanna Is Influenced By Spatial Extent and Environmental Variation. Ecography:no-no.}

\hypertarget{citeproc_bib_item_118}{Wilson, K. B., D. D. Baldocchi, and P. J. Hanson. 2000. Spatial and Seasonal Variability of Photosynthetic Parameters and Their Relationship To Leaf Nitrogen in a Deciduous Forest. Tree Physiology 20:565–578.}

\hypertarget{citeproc_bib_item_119}{Wirth, C., and J. W. Lichstein. 2009. The Imprint of Species Turnover on Old-Growth Forest Carbon Balances - Insights From a Trait-Based Model of Forest Dynamics. Ecological Studies:81–113.}

\hypertarget{citeproc_bib_item_120}{Wohlfahrt, G., M. Bahn, E. Haubner, I. Horak, W. Michaeler, K. Rottmar, U. Tappeiner, and A. Cernusca. 1999. Inter-Specific Variation of the Biochemical Limitation To Photosynthesis and Related Leaf Traits of 30 Species From Mountain Grassland Ecosystems Under Different Land Use. Plant, Cell and Environment 22:1281–1296.}

\hypertarget{citeproc_bib_item_121}{Wright, I. J., D. D. Ackerly, F. Bongers, K. E. Harms, G. Ibarra-Manriquez, M. Martinez-Ramos, S. J. Mazer, H. C. Muller-Landau, H. Paz, N. C. A. Pitman, and et al. 2006. Relationships Among Ecologically Important Dimensions of Plant Trait Variation in Seven Neotropical Forests. Annals of Botany 99:1003–1015.}

\hypertarget{citeproc_bib_item_122}{Wright, I. J., P. B. Reich, M. Westoby, D. D. Ackerly, Z. Baruch, F. Bongers, J. Cavender-Bares, T. Chapin, J. H. C. Cornelissen, M. Diemer, and et al. 2004. The Worldwide Leaf Economics Spectrum. Nature 428:821–827.}

\hypertarget{citeproc_bib_item_123}{Wright, J. P., and A. Sutton-Grier. 2012. Does the Leaf Economic Spectrum Hold Within Local Species Pools Across Varying Environmental Conditions? Functional Ecology 26:1390–1398.}

\hypertarget{citeproc_bib_item_124}{Wright, S. J., K. Kitajima, N. J. B. Kraft, P. B. Reich, I. J. Wright, D. E. Bunker, R. Condit, J. W. Dalling, S. J. Davies, S. Díaz, and et al. 2010. Functional Traits and the Growth-Mortality Trade-Off in Tropical Trees. Ecology 91:3664–3674.}

\hypertarget{citeproc_bib_item_125}{Xu, L., and D. D. Baldocchi. 2003. Seasonal Trends in Photosynthetic Parameters and Stomatal Conductance of Blue Oak (Quercus douglasii) Under Prolonged Summer Drought and High Temperature. Tree Physiology 23:865–877.}

\hypertarget{citeproc_bib_item_126}{Yguel, B., R. Bailey, N. D. Tosh, A. Vialatte, C. Vasseur, X. Vitrac, F. Jean, and A. Prinzing. 2011. Phytophagy on Phylogenetically Isolated Trees: Why Hosts Should Escape Their Relatives. Ecology Letters 14:1117–1124.}

\hypertarget{citeproc_bib_item_127}{Zapata-Cuartas, M., C. A. Sierra, and L. Alleman. 2012. Probability Distribution of Allometric Coefficients and Bayesian Estimation of Aboveground Tree Biomass. Forest Ecology and Management 277:173–179.}
\end{hangparas}
\end{document}
